\chapter{Marco Experimental}


En este capítulo, son descritos los conjuntos de datos sobre los que se evaluará el desempeño de la estrategia de análisis modular propuesta para abordar el perfilado de autores. Luego se exponen las metricas empleadas y finalmente se presentan los resultados experimentales asociados a cada módulo (i.e., Codificador y Clasificador) de manera independiente y a las combinaciones de los mismos.
	
\section{Tareas}
	 
	 Las colecciones de datos existente relacionadas con tareas de AP mayormente se encuentran anotadas a nivel de perfil y con información relativa a tareas especificas. Para evaluar los modelos propuestos, siguiendo este esquema, en este trabajo se introducen los conjuntos de datos de las tareas compartidas en las ediciones de 2019-2021 de PAN, cada una de estas propuestas con un enfoque multilingüe, analizando el problema para perfiles de usuarios del idioma ingles y español.
	 
	 \subsection{Bots and Gender Profiling at PAN 2019}
	 
	 En la tarea Bots and Gender Profiling, dado conjunto de exactamente 100 posts  pertenecientes a un perfil de usuario de Twitter y teniendo en cuenta que no existe ningún tipo anotación a nivel de tweets, debe determinarse si este corresponde a un bot o a un ser humano y para el segundo caso predecirse el genero sexual de la persona.
	 \\
	 Para evaluar el desempeño de los modelos propuestos, en nuestro trabajo se prestará atención solamente al problema de discernir entre perfiles de usuarios humano hombres o mujeres. El conjunto de datos de perfiles humanos está construido a partir de cuentas de usuarios tomadas de corpus creados en ediciones previas de la tarea de perfilado de PAN \citep{rangel2017overview, rangel2018overview}. Además los datos se encuentran distribuidos uniformemente tanto para el idioma ingles como para el español, entre las clases `\textit{Male}' y `\textit{Female}' como puede ser observado en la \tablename~\ref{pan19data}.	 
	 		\begin{table}[thb!]
			 	\begin{center} 					 		
			 		\begin{tabular}{lcccccc} 
			 			\specialrule{.1em}{.05em}{.05em}
			 			 \multirow{2}{*}{}&\multicolumn{3}{c}{(EN) Ingles}&\multicolumn{3}{c}{(ES) Español}\\	 			\cline{2-7}
			 			&~~Female~~&~~Male~~&~~Total~~ &~~Female~~ &~~Male~~&~~Total~~\\
			 			\specialrule{.1em}{.05em}{.05em} 
			 			Training & 1030&1030&2060&750&750&1500\\
			 			Test  &660&660&1320&450&450&900\\
			 			\cline{1-7}
			 			Total &1690&1690&3380&1200&1200&2400\\
			 			\specialrule{.1em}{.05em}{.05em} 
			 		\end{tabular}
			 		\caption[Corpus Profiling PAN 2019]{Distribución de los datos para la tarea Bots and Gender Profiling at PAN 2019}	
			 		\label{pan19data}	
			 	\end{center}
			 \end{table}	
		 \\
	 \subsection{Profiling Fake News Spreaders on Twitter at PAN 2020}
	 
	 Profiling Fake News Spreaders on Twitter, dentro de PAN 2020, introduce el análisis de rasgos de la personalidad de los autores, proponiendo la tarea de discriminar entre usuarios de Twitter que han compartido noticias falsas de aquellos que nunca lo han hecho, basándose en un conjunto de 100 tweets (carentes de algún tipo de anotación) tomados de su perfil.\\
	 El corpus propuesto por los organizadores de la tarea \citep{francisco_rangel_2020_4039435}, fue construido seleccionando de sitios web  \textit{fact-checking } (comprobadores de hechos) noticias etiquetadas como falsas, luego mediante la búsqueda de tweets relacionados con estas \textit{fake news}, se identificaron a sus correspondientes usuarios como ejemplos positivos de \textit{fake news spreaders} (faker) tomando aquellos con un mayor número de noticias falsas compartidas y teniendo en cuenta que el contenido del tweet no fuera para desmentir la noticia falsa.	 En el caso de que el usuario no hubiera compartido información relacionada con las noticias falsas identificadas, este fue etiquetado como \textit{real news spreader} (no faker).
	 
	 El la \tablename~\ref{pan20data} se muestra la distribución uniforme de los perfiles para los idiomas español e ingles en los que fue compartida la tarea.	 
	 \\
	 	\begin{table}[thb!]
	 	\begin{center} 					 		
	 		\begin{tabular}{lcccccc} 
	 			\specialrule{.1em}{.05em}{.05em}
	 			\multirow{2}{*}{}&\multicolumn{3}{c}{(EN) Ingles}&\multicolumn{3}{c}{(ES) Español}\\	 			\cline{2-7}
	 			&~~faker~~&~~no faker~~&~~Total~~ &~~faker~~ &~~no faker~~&~~Total~~\\
	 			\specialrule{.1em}{.05em}{.05em} 
	 			Training & 150&150&300&150&150&300\\
	 			Test  &100&100&200&100&100&200\\
	 			\cline{1-7}
	 			Total &250&250&500&250&250&500\\
	 			\specialrule{.1em}{.05em}{.05em} 
	 		\end{tabular}
	 		\caption[Corpus Profiling PAN 2020]{Distribución de los datos para la tarea Profiling Fake News Spreaders on Twitter at PAN 2020}	
	 		\label{pan20data}	
	 	\end{center}
	 \end{table}	
	 
	 \subsection{Profiling Hate Speech Spreaders on Twitter at PAN 2021}
	 
	 Para esta tarea dado un perfil de usuario, debía ser determinado cuando este corresponde a un autor que ha difundido en el pasado un discurso de odio teniendo en cuenta 200 tweets tomados de su perfil.\\
	 El corpus propuesto por los organizadores fue construido considerando usuarios que han empleado palabras con cierto nivel de toxicidad fundamentalmente relacionadas con la misoginia y xenofobia, ademas se inspeccionaron cuentas de usuarios conocidos como \textit{haters} con apariciones en reportes, asi como su red i.e., followers. Luego para estos usuarios identificados, fueron anotados manualmente los tweets que comunicaban un discurso de odio y finalmente fueron clasificados como \textit{hate speech spreaders} aquellos usuarios con mas de 10 de estos posts. El conjunto de datos esta distribuido uniformemente para las clases divulgador de discurso de odio (\textit{hater})  y no divulgador de discurso de odio (\textit{no hater}) como se muestra en la \tablename~\ref{pan21data}
	 \\	 
	 	\begin{table}[thb!]
		 	\begin{center} 					 		
		 		\begin{tabular}{lcccccc} 
		 			\specialrule{.1em}{.05em}{.05em}
		 			\multirow{2}{*}{}&\multicolumn{3}{c}{(EN) Ingles}&\multicolumn{3}{c}{(ES) Español}\\	 			\cline{2-7}
		 			&~~hater~~&~~no hater~~&~~Total~~ &~~hater~~ &~~no hater~~&~~Total~~\\
		 			\specialrule{.1em}{.05em}{.05em} 
		 			Training & 150&150&300&150&150&300\\
		 			Test  &100&100&200&100&100&200\\
		 			\cline{1-7}
		 			Total &250&250&500&250&250&500\\
		 			\specialrule{.1em}{.05em}{.05em} 
		 		\end{tabular}
		 		\caption[Corpus Profiling PAN 2021]{Distribución de los datos para la tarea Profiling Hate Speech Spreaders on Twitter at PAN 2021}	
		 		\label{pan21data}	
		 	\end{center}
		 \end{table}	
	 
	 \section{Resultados Experimentales}
	 
	 La robustez de los sistemas modulares descansan en el optimo aprendizaje de cada uno de los módulos de manera independiente, en nuestro caso del Codificador y el Clasificador. De manera general para este trabajo se introduce una tarea intermedia semisupervisada para entrenar los modelos Codificadores \footnote{El objeto de predicción de esta tarea semisupervisada, estuvo condicionado por la carencia de anotación a nivel de tweet en cada uno de los corpus y se relaciona con la pertenencia o no a determinado tipo de perfil.}, de esta forma los mismos aprenden relaciones del lenguaje dentro de cada tweet sobre la tarea en cuestión.\\
	 Para medir el cuan efectivos resulta cada modelo empleamos las métricas \textit{accuracy} (\ref{accuracy}) teniendo en cuenta los datos clasificados clasificados correctamente para los ejemplos positivos y negativos (TP y TN) y los clasificados de manera errónea (FP y FN).
	 %% y\textit{ recall} (\ref{recall}), esta última para los modelos de clasificación sobre la clase positiva en las tareas en las que es necesario la correcta identificacion de una clase sobre otra, i.e., clase positiva (hater) en \textit{Profiling Hate Speech Spreaders on Twitter}  y clase positiva (faker) en \textit{Profiling Fake News Spreaders on Twitter}.
	 
	 \begin{align}
	 	acc=~& \frac{TP + TN}{TP + FP + TN + FN}\label{accuracy}
	 \end{align}\\%recall =~& \frac{TP}{TP + FN}\label{recall}
 
 	\subsection{Codificador CNN - LSTM}
 	
 	Para el análisis secuencial de los tweets de manera independiente, el primer modelo en analizar será CNN - LSTM. Este combina relaciones espaciales detectadas por una 1D-CNN mediante una red recurrente, la cual expresa como relaciones a largo plazo la información de los n-gramas condensados por la CNN como se muestra en la \figurename~\ref{cnn_lstm}.   Sin embargo para nuestro enfoque es añadido el análisis en paralelo de los niveles sintácticos de palabra y caracteres.  	\\
 	
 	Debido a la finitud de la cardinalidad de los diccionarios en los \textit{embeddings de palabras}, la cual esta dada por la memoria y tiempo disponible para entrenar un modelo de \textit{embedding}, es recurrente que al analizar textos informales como los tweets se pierda información aportada por elementos que no están presentes en el diccionario, pero que si contienen información semántica y de estilo, ejemplo de ello son las palabras con elongación de caracteres (e.g., holaaaa) y/o \textit{typos}. Por ello empleando la arquitectura descrita en la Sección~\ref{cnn-lstm} introducimos el análisis a nivel de caracteres el cual es combinado con el de palabras a través de una fusión multi-fuente directa\footnote{Simple concatenación de los vectores de cada una de las fuentes, i.e., arquitectura a nivel de palabra y de caracteres} enviada a una capa densa de neuronas.
 	\\\\
 	Para entrenar este modelo teniendo en cuenta cada uno de los corpus y tareas de AP en las cuales se medirá su desepeño, se analiza la pertenencia o no de un tweet a clase u otra de perfiles, e.g., si el tweet pertenece a la cuenta de un hombre o una mujer, \textit{hater }o \textit{no hater}, etc. De esta manera, se captura no solo las relaciones inherentes al lenguaje presente en el corups frente a la tarea, si no que se aprende como estas relaciones existen atendiendo a un tipo de perfil u otro.
 	\\
 	En el proceso de entrenamiento, el conjunto de datos de entrenamiento de cada uno de los corpus es dividido a una razón de 1:5 en un subconjunto de validación y otro de entrenamiento, de manera que se preserve la distribución uniforme de ejemplos positivos y negativos en cada subconjunto.
 	Para la arquitectura es añadida una neurona en el tope del modelo encargada de responder la probabilidad de que el tweet pertenezca a una clase de cuenta y los pesos del modelo son ajustados por el oprimizador de Estimación Adaptativa del Momento (\textit{Adaptive Moment Estimation}) Adam \citep{DBLP:journals/corr/KingmaB14} empleando la función de perdida de \textit{Cross-Entropy}.  	\\\\
 	Durante los experimentos exploramos como afectaba la variación hiper-parametro del coeficiente de aprendizaje $\alpha \in \{1e\text{-}3,~15e\text{-}4,~2e\text{-}3,~3e\text{-}3\}$  en el optimizador de Adam a través de cada tarea, i.e., Bots and Gender Profiling at PAN 2019 (\textit{gender}), Profiling Fake News Spreaders on Twitter (\textit{faker}) y Profiling Hate Speech Spreaders on Twitter (\textit{hater}) como se muestra en la \tablename~\ref{cnn_lstm_train} en términos de \textit{accuracy}. Resultando para cada modelo en cada variación de lenguaje valores óptimos de  $alpha$ entre $1e\text{-}3\text{ y }2e\text{-}3$.
 	\\
 	\begin{table}[thb!]
 		\begin{center} 					 		
 			\begin{tabular}{l|cccc|cccc} 
 				\specialrule{.1em}{.05em}{.05em}
 				\multirow{2}{*}{Tarea}&\multicolumn{4}{c}{(EN) Ingles}&\multicolumn{4}{c}{(ES) Español}\\	 			\cline{2-9}
 				&~~$1e\text{-}3$~~&~~$15e\text{-}4$~~&~~$2e\text{-}3$~~ &~~$3e\text{-}3$~~ &~~$1e\text{-}3$~~&~~$15e\text{-}4$~~&~~$2e\text{-}3$~~ &~~$3e\text{-}3$\\
 				\specialrule{.1em}{.05em}{.05em} 
 				gender & \textbf{0.684}&0.68&0.681&0.677&\textbf{0.698}&0.693&0.696&0.691\\
 				faker  &0.734&\textbf{0.738}&0.727&0.726&0.741&\textbf{0.742}&0.736&0.730\\
 				hater &0.696&0.692&\textbf{0.698}&0.694&0.611&0.608&\textbf{0.611}&0.584\\
 				\specialrule{.1em}{.05em}{.05em} 
 			\end{tabular}
 			\caption[CNN-LSTM $\alpha$ tuning ]{Resultados del entrenamiento para el modelo CNN-LSTM en cada tarea según el coeficiente de aprendizaje $\alpha$}	
 			\label{cnn_lstm_train}
 		\end{center}
 	\end{table}	
 
 	\subsection{Codificador Transformer}
 	
	Para el caso de los codificadores basados en arquitecturas transformers, nuestra propuesta emplea un modelo base para cada idioma, a pesar de que ambos tienen la misma configuracion de BERT-base \citep{DBLP:journals/corr/abs-1810-04805} y son preentrenados siguiendo la misma estrategia de RoBERTa \citep{liu2019roberta}, i.e., enmascarando de manera aleatoria el 15\% de las palabras de la entrada del modelo en la tarea de modelado de lenguaje enmascarado (Masked Language Modeling MLM).
	\\ 
	Para el idioma español se emplea el modelo BETO \citep{CaneteCFP2020} y para ingles BERTweet \citep{bertweet}, ambos de la biblioteca Transformers de HuggingFace \footnote{\url{https://huggingface.co/transformers}}. BERTweet fue preentrenado con un corpus de tweets en ingles, mientras que para BETO , se emplearon textos de otras fuentes como wikis en español, OpenSubtitles y ParaCrawl \footnote{\url{https://github.com/josecannete/spanish-corpora}}.
	\\\\
	En el proceso de finetuning al igual que con la arquitectura CNN - LSTM estos modelos fueron refinados atendiendo a la tarea de predecir cuando un tweet pertenece o no a una clase, siguiendo la estrategia descrita en la Sección~\ref{ref_trans} de aplicar un coeficiente $\alpha$ dinámico a medida que se profundiza en la red, el cual ha sido empleado con resultados alentadores por \citep{palomino-ochoa-luna-2020-palomino,Sem}. Los parámetros de los modelos fueron ajustados empleando el optimizador RMSprop \citep{hinton2012lecture} con un descenso $\delta$ de learning rate lineal por cada epoch, bajo la función de perdida de \textit{Cross-Entropy} y con el mismo esquema de partición de los datos 1:5.	\\
	En la \tablename~\ref{transf_finet} se muestra el \textit{accuracy} obtenido para cada tarea en el proceso de refinado, donde se hace evidente un mejor desempeño respecto a los resultados alcanzados con la arquitectura CNN - LSTM de la \tablename~\ref{cnn_lstm_train}. 
		\begin{table}[thb!]
		\begin{center} 					 		
			\begin{tabular}{lcc} 
				\specialrule{.1em}{.05em}{.05em}
				Tarea&EN& ES\\	
				\specialrule{.1em}{.05em}{.05em} 
				gender & 0.84&0.767\\
				faker  &0.65&0.726\\
				hater &0.836&0.805\\
				\specialrule{.1em}{.05em}{.05em} 
			\end{tabular}
			\caption{Finetuning Transformers en cada tarea con $\alpha=1e\text{-}5,~\delta = 2e\text{-}5, ~\lambda=0.1$}\label{transf_finet}	
		\end{center}
	\end{table}		\\
	En particular en ninguna de las dos variantes de idioma sobre la tarea faker, existe tal mejora. Teniendo en cuenta que para la tarea de determinar si un tweet corresponde a un autor \textit{faker} o no, la existencia de sesgo en la representación dado por la información semántica del mensaje podría afectar el rendimiento del modelo, podemos hipotetizar que esta es la causa de la clasificación errónea de algunos tweets. \\
	Por otro lado el desempeño alcanzado en las tareas \textit{gender} y \textit{hater} es razonable dada la forma en que el mecanismo de atención empleado relaciona la información y a la cantidad de parámetros de estos modelos, lo cual permite que cada neurona  preste atención a estructuras particulares del lenguaje. Esto último pudiera resultar en un sobreajuste a los datos de entrenamiento de no ser por la enorme cantidad datos de diversas fuentes con las que fueron preentrenados y el $\alpha$ suave empleado en el proceso de \textit{finetuning}. 
	
	
	\subsection{Clasificador Att-LSTM}
	
	Al igual que en los módulos de codificación, este modelo fue entrenado de manera independiente para cada idioma, mediante el optimizador de Adam, de manera que exploramos para cada representación como su combinación con los rasgos de estilo (Sección~\ref{syle_feat}) podría afectar el desempeño del modelo dado un coeficiente de aprendizaje $\alpha$. \\
	En la \tablename~\ref{att-lstm_train} se muestran los resultados de dicho entrenamiento, evaluado sobre el conjunto de datos de prueba propuesto por los organizadores de cada tarea, donde se puede observar que las modelaciones de los perfiles tomando como elementos de la secuencia tanto de un codificador como del otro arrojaron resultados alentadores. Además en las tareas relacionadas directamente con dispositivos comunicacionales, fundamentalmente en la tarea \textit{hater}, los rasgos de estilo tuvieron un impacto positivo en el \textit{accuracy} del modelos, sin embargo, para la tarea \textit{gender} estos rasgos de estilo al parecer resultaron ruidosos o no  tuvieron influencia en el modelado del perfil por parte de Att-LSTM al emplear ninguna de las dos representaciones aprendidas por los modelos de DL.
	
	
	 	\begin{table}[thb!]
		\begin{center} 					 		
			\begin{tabular}{l|c|cccc|cccc} 
				\specialrule{.1em}{.05em}{.05em}
				\multirow{2}{*}{Tarea}&\multirow{2}{*}{$\alpha$}&\multicolumn{4}{c}{(EN) Ingles}&\multicolumn{4}{c}{(ES) Español}\\	 			\cline{3-10}
				&&C1& C2 &C2-STY &C1-STY&C1& C2 &C2-STY &C1-STY\\
				\specialrule{.1em}{.05em}{.05em} 
				\multirow{4}{*}{gender} &$5e\text{-}4$&0.797&0.758&0.761&0.800&0.630&0.648&0.650&0.630 \\
				&$1e\text{-}3$&0.795&\textbf{0.803}&0.789&0.797&0.632&\textbf{0.693}&0.660&0.639 \\
				&$2e\text{-}3$&0.797&0.802&0.782&0.793&0.635&0.683&0.621&0.637\\
				&$3e\text{-}3$&0.801&0.753&0.779&0.802&0.632&0.667&0.650&0.632 \\
				 \hline
				\multirow{4}{*}{faker} &$5e\text{-}4$&0.690&0.645&0.635&0.694&0.785&0.805&0.800&0.780 \\
				&$1e\text{-}3$&0.680&0.650&0.645&0.700&0.784&0.805&0.805&0.784\\
				&$2e\text{-}3$&0.690&0.660&0.665&\textbf{0.704}&0.775&0.810&0.815&0.795\\
				&$3e\text{-}3$&0.685&0.645&0.655&0.695&0.770&0.810&\textbf{0.820}&0.779 \\
				\hline
				\multirow{4}{*}{hater} &$5e\text{-}4$&0.650&0.700&\textbf{0.740}&0.660&0.800&0.789&0.798&0.790 \\
				&$1e\text{-}3$&0.650&0.730&0.700&0.660&0.790&0.800&0.798&\textbf{0.810} \\
				&$2e\text{-}3$&0.700&0.709&0.710&0.670&0.780&0.800&\textbf{0.810}&\textbf{0.810}\\
				&$3e\text{-}3$&0.649&0.719&\textbf{0.740}&0.670&0.790&0.790&0.790&\textbf{0.810}\\
				\specialrule{.1em}{.05em}{.05em} 
			\end{tabular}
			\caption[Entrenamiento de Att-LSTM]{Entrenamiento del modelo Att-LSTM basado en el análisis secuencial del perfil para la combinación de rasgos de los modelos: (i) codificador CNN - LSTM (C1), (ii) codificador transformers y rasgos de estilo (STY)}	
			\label{att-lstm_train}
		\end{center}
	\end{table}	
	
	\subsection{Clasificador SGN}
	
	
	
	
	EXPLICAR COMO FUSIONASTE LOS RASGOS DE ESTILO  EN LA SECCION DE ARRIBA
	
	QUIENES SON USTEDES PABLO VITART??
	
	
	
	
	