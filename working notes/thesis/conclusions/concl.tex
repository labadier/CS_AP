\addcontentsline{toc}{chapter}{Conclusiones y Trabajo Futuro}
\chapter*{Conclusiones y Trabajo Futuro}

En este trabajo se abordó la tarea de perfilado de autores con un enfoque multilingüe, analizando perfiles en los idiomas español e inglés, mediante la propuesta de una arquitectura modular que divide el problema de clasificar un perfil de usuario de Twitter en i) determinar una representación para cada tweet de manera individual en un espacio latente, empleando una de dos técnicas; la combinación del análisis a nivel de palabra y caracteres a través de redes neuronales convolucionales y de memoria a largo-corto plazo, o modelos basados en la arquitectura transformers; ii) modelar el perfil de usuario para clasificarlo empleando una de dos interpretaciones: la primera consiste en tomar el conjunto de tweets dentro del perfil como una secuencia y la otra consiste en interpretar el perfil como un grafo completo donde todos los nodos se relacionan entre sí y mediante una red neuronal convolucional espectral aprender como la representación de un tweet pertenece a su contexto para clasificar el grafo.
\\
Además, se propuso el uso de un conjunto de rasgos de estilos para estudiar cómo estos influyen en la precisión de los modelos de Deep Learing propuestos para clasificar los perfiles de usuario y se introdujo una variación al Método de los Impostores empleado anteriormente en tareas de verificación de autoría. Con esta modificación el método fue capaz de manejar representaciones determinadas a partir de modelos basados en DL y se convirtió la tarea de clasificación de un perfil a una tarea de verificación, donde a través de un conjunto de prototipos se asignó una clase u otra a dicho perfil.  
\\
Los resultados experimentales conducidos mostraron la robustez de los modelos independientemente del lenguaje, frente a las propuestas de Deep Learning del estado del arte; en especial la modelación del perfil basada en grafos empleando las representaciones latentes de las arquitecturas transformers, obtuvo resultados importantes. Por otra parte se identificó la influencia positiva de los rasgos de estilo propuestos para tareas que no involucraban dispositivos del lenguaje complejos como el odio y la ofensa, sino que dependían más de la forma en la que se comunica la información; tal es el caso de la tarea de determinar el género sexual o la tendencia a publicar noticias falsas por parte de los usuarios.
\\\\
Teniendo en cuenta todo esto, podemos afirmar que nuestro trabajo ha cumplido con sus objetivos y pretendemos introducir los siguientes elementos para mejorar el \textit{accuracy} de nuestra arquitectura modular:
\begin{itemize}
	\item Puesto que el modelado basado en grafos parece ser una técnica prometedora para expresar aquellas relaciones no estructuradas como las existentes entre los posts, se pretende analizar y construir una representación que capture relaciones más fuertes considerando las predicciones que produzcan los modelos codificadores sobre los tweets y la similitud entre los mismos.
	\item Como alternativa a una construcción más costosa en espacio de memoria y tiempo de una estructura de grafos más compleja, se puede seguir considerando el grafo completo con una función de agregación para el paso de mensajes, basada en modelos de atención que permita al SGN discernir cuáles tweets de una vecindad son más importantes a la hora de determinar la representación contextual.
	\item Para nuestra modificación del Método de los Impostores, explorar métodos para la selección de prototipos que permitan evitar la selección del hiperparámetro empleado \textit{p} y hacer de esta arquitectura un modelo End-to-End, así como construir un sub-modelo de aprendizaje de métrica que sustituya a la función coseno empleada en nuestra propuesta.
\end{itemize}
\clearpage